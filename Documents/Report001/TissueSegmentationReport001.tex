\documentclass{InsightArticle}

\usepackage[dvips]{graphicx}
\usepackage{color}
\usepackage{listings}

\definecolor{listcomment}{rgb}{0.0,0.5,0.0}
\definecolor{listkeyword}{rgb}{0.0,0.0,0.5}
\definecolor{listnumbers}{gray}{0.65}
\definecolor{listlightgray}{gray}{0.955}
\definecolor{listwhite}{gray}{1.0}


%%%%%%%%%%%%%%%%%%%%%%%%%%%%%%%%%%%%%%%%%%%%%%%%%%%%%%%%%%%%%%%%%%
%
%  hyperref should be the last package to be loaded.
%
%%%%%%%%%%%%%%%%%%%%%%%%%%%%%%%%%%%%%%%%%%%%%%%%%%%%%%%%%%%%%%%%%%
\usepackage[dvips,
bookmarks,
bookmarksopen,
backref,
colorlinks,linkcolor={blue},citecolor={blue},urlcolor={blue},
]{hyperref}


\title{Tissue Segmentation in Testis Image Analysis}


%
% NOTE: This is the last number of the "handle" URL that
% The Insight Journal assigns to your paper as part of the
% submission process. Please replace the number "1338" with
% the actual handle number that you get assigned.
%
\newcommand{\IJhandlerIDnumber}{3063}

\lstset{frame = tb,
       framerule = 0.25pt,
       float,
       fontadjust,
       backgroundcolor={\color{listlightgray}},
       basicstyle = {\ttfamily\footnotesize},
       keywordstyle = {\ttfamily\color{listkeyword}\textbf},
       identifierstyle = {\ttfamily},
       commentstyle = {\ttfamily\color{listcomment}\textit},
       stringstyle = {\ttfamily},
       showstringspaces = false,
       showtabs = false,
       numbers = left,
       numbersep = 6pt,
       numberstyle={\ttfamily\color{listnumbers}},
       tabsize = 2,
       language=[ANSI]C++,
       floatplacement=!h
       }

\release{1.10}

\author{Luis Ibanez$^{1}$, Wes Turner$^{1}$, Fanny Odet$^{2}$, Deborah O'Brian$^{2}$}
\authoraddress{$^{1}$Kitware Inc., Clifton Park, NY\\
               $^{2}$Cell and Developmental Biology Department, UNC, Chapel Hill, NC}

\begin{document}


%
% Add hyperlink to the web location and license of the paper.
% The argument of this command is the handler identifier given
% by the Insight Journal to this paper.
%
\IJhandlefooter{\IJhandlerIDnumber}


\ifpdf
\else
   %
   % Commands for including Graphics when using latex
   %
   \DeclareGraphicsExtensions{.eps,.jpg,.gif,.tiff,.bmp,.png}
   \DeclareGraphicsRule{.jpg}{eps}{.jpg.bb}{`convert #1 eps:-}
   \DeclareGraphicsRule{.gif}{eps}{.gif.bb}{`convert #1 eps:-}
   \DeclareGraphicsRule{.tiff}{eps}{.tiff.bb}{`convert #1 eps:-}
   \DeclareGraphicsRule{.bmp}{eps}{.bmp.bb}{`convert #1 eps:-}
   \DeclareGraphicsRule{.png}{eps}{.png.bb}{`convert #1 eps:-}
\fi


\maketitle


\ifhtml
\chapter*{Front Matter\label{front}}
\fi


\begin{abstract}
\noindent
This document describes a contribution to the Insight Toolkit intended to
perform tissue segmentation in micrographs from testis cross-sections from
knockout mice.

This paper is accompanied with the source code, input data, parameters and
output data that we used for validating the algorithm described in this paper.
This adheres to the fundamental principle that scientific publications must
facilitate \textbf{reproducibility} of the reported results.
\end{abstract}

\tableofcontents

\section{Introduction}

The Collaborative Cross (CC) is a mammalian reference population designed to
provide a cross-disciplinary platform for systems genetics that models the
heterogeneous human population.  Specifically, the CC is a large panel of
recombinant inbred (RI) mouse lines derived from eight genetically diverse
founder strains.  RI panels are powerful tools for complex trait analysis
needed to understand disease susceptibility and progression. More than $70\%$ of
the initial lines have become extinct during inbreeding, and our initial
results indicate that $~50\%$ of the extinction cases are due to male
infertility.  We are using this resource to first identify loci, and eventually
genes, associated with male infertility and other male reproductive traits.
Reproductive phenotyping of males from each extinct line is our first step in
defining the genetic basis of the observed infertility.

Testis histology is an important component of our phenotypic analysis, and is
also widely used to assess causes of infertility in knockout mice.  In the CC
project, we already have >900 images of testis cross-sections stained with PAS
and hematoxylin.  These are composite images generated with MetaMorph software
from 20-100 color micrographs (24 bit) captured with brightfield illumination
using a 20x objective. Our initial goals for image analysis are to:

\begin{itemize}
\item identify seminiferous tubule cross-sections (smaller, mostly spherical or
oval structures that fill most of the image).  Each tubule is surrounded by a
magenta border. The epithelium within each tubule has a speckled appearance
with nuclei (blue/purple to magenta) of various shapes and sizes on a lighter
pink background.  Most tubules have a clear lumen with no nuclei.  There are
other cells, blood vessels and spaces between the tubules.
\item count the tubule cross sections in each image
\item determine the diameter of each tubule
\item identify the clear lumen present in the center of most tubules
\item determine the proportion of each tubule that is filled with cells.
Either epithelial height or area would be useful, since both are reduced when
germ cells are prematurely lost during differentiation.
\end{itemize}

A more automated approach for any of these steps would facilitate the
collection of quantitative data preferred for genetic analyses.  Our images
display a broad continuum from normal samples (tubules with a relatively
uniform diameter and epithelial height) to samples that have lost all
developing germ cells (smaller tubules with no clear epithelium).  Many of our
images contain a mixture of normal tubules and abnormal tubules.  I’ve uploaded
three images to illustrate the variability between samples. Any progress
towards an automated analysis would help a great deal, even if we can develop a
routine that correctly identifies a significant fraction of the tubules and can
be edited by manual identification.


\section{Statistical Classification}


\clearpage
\begin{figure}
\center
\begin{tabular}{c}
\includegraphics[width=\textwidth]{../../Testing/Temporary/ernou_tor5564m12_small.png}
\end{tabular}
\itkcaption[Original Image]{Cropped section of original image}
\label{fig:OriginalImage}
\end{figure}


\clearpage
\begin{figure}
\center
\begin{tabular}{c}
\includegraphics[width=\textwidth]{../../Testing/Temporary/ernou_tor5564m12_small_classified_colors.png}
\end{tabular}
\itkcaption[Classified Image]{Classified image}
\label{fig:OriginalImage}
\end{figure}


\clearpage
\begin{figure}
\center
\begin{tabular}{c}
\includegraphics[width=\textwidth]{../../Testing/Temporary/ernou_tor5564m12.png}
\end{tabular}
\itkcaption[Original Image]{Cropped section of original image}
\label{fig:OriginalImage}
\end{figure}


\clearpage
\begin{figure}
\center
\begin{tabular}{c}
\includegraphics[width=\textwidth]{../../Testing/Temporary/ernou_tor5564m12_classified_colors.png}
\end{tabular}
\itkcaption[Classified Image]{Classified image}
\label{fig:OriginalImage}
\end{figure}




%%%%%%%%%%%%%%%%%%%%%%%%%%%%%%%%%%%%%%%%%
%
%  Insert the bibliography using BibTeX
%
%%%%%%%%%%%%%%%%%%%%%%%%%%%%%%%%%%%%%%%%%

\bibliographystyle{plain}
\bibliography{InsightJournal}


\end{document}
